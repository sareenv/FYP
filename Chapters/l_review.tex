\subsection{Convential Diagnosis Methods}

The most common conventional diagnosis method of detection involves
using ABCD rule which considers the Asymmetry, Border irregularity, Colour
irregularities,  Darmascopic structures respectively of common pigmented skin
Lesions \citep*{LOESCHER2013170}. The above method of analysis is performed on 
dermoscopic images of pigmented skin lesions.  Dermatoscopy is 
non-invasive microscopic imaging of pigmented skin lesions which provides clear imaging to perform 
proper analysis on pigmented skin lesions \citep*{LOESCHER2013170}. 
Furthermore, the result of dermatoscopic images is examined 
by dermatologists to classify the pigmented skin lesion.  \textbf{SOME SAYS ....}

\subsection{Support Vector Based Machine}
Thompson Felsia and Jeyakumar proposed research in 2017 on 
support vector machine based classifier to detect multi-lesions skin cancer by analysing pigmented skin lesions with an accuracy of 86.37 percent.
The proposed investigation with SVM based classifier has performed image segmentation using SRM (support region merging) algorithm. Furthermore, it employs SURF (speed up robust features) to find the region 
of interest for feature extraction to get optimal classification performance based on vector-based technique \citep*{thompson2017vector}. 
However, the research does not include image augmentation which generalises the predictions accurate to test in 
real-world environment. The research papers mention that support vector machine for automated classification of pigmented skin lesions is sensitive to the artefacts and can 
potentially increase the false positives which mean that predicted result for analysis was wrong positive prediction instead of an actual negative result. The investigation will perform image augmentation to generate random 
samples of images with different rotation angle and flipped images will be used to train and test the model to generalise the overall performance.

\subsection{Border Detection Based System}
Rahil Garnavi and his other co-researchers purposed research based on a state of the art border detection method combined with the colour space analysis and clustering-based histogram hybrid thresholding to classify pigmented skin lesions.
 The research was primarily focused on the research was to develop the hair removal mechanism to perform colour channels transformation. Furthermore, for all the image channels the noise reduction 
and clustering-based histogram thresholding were performed for optimal border detection. The predicted outcomes of novel broder detection system were compared with the 
borders detected by the actual dermatologists on a sample of 
dermoscopic pigmented skin lesions to understand the reliability of the 
system \citep*{GARNAVI2011105}.However, the system was only tested on a data sample set of 
30 dermoscopic images and four sample sets of dermatologist hand-drawn images were used as ground truth to compare the results. The system was tested on overall 85 dermoscopic images.
Border detection can be used to analyse the pigmented skin lesions but convolutional networks have the potential to find more data patterns in the images to minimise the cost function 
using the backpropagation algorithm. The current research will 
employe basic image segmentation based on the binary threshold 
algorithm as an experiment to help network detecting more accurate
borders of pigmented skin lesions.

\subsection*{Deep Feature to classify Pigmented Skin lesions}

In 2016, a research paper from Simon Fraser University’s computer science and medical image analysis lab had researched using 
deep residual network architecture with ten labelled 
classes of pigmented skin lesions. The research was based on very 
deep convolutional network architecture with the accuracy 
of 85.8 percent in classifying five distinct classes and
81 percent in classifying 10 classes of pigmented skin lesions
\citep*{7493528}. Although the performance of the overall convolutional network was accurate, the training and testing data were limited to 13,00 overall images of 10 distinct classes.
 However, In the current research project, the classes of labelled images will be five and around 9,000 overall images will be used during the investigation.
 Estimated 80 percent of data will be consumed for training the model, and the rest of the label images will be used as validation and testing datasets 
 to evaluate the performance of the model. 
 Research is also consuming such artificial neural network-based technologies to various areas of investigations.