\subsection{Convential Diagnosis Methods}
Dermatoscopy is non-invasive microscopic imaging of pigmented skin lesions. The results of dermoscopy are further accessed by the dermatologists to classify the pigmented skin lesion. The current diagnosis method of detection involves using ABCD rule which considers the Asymmetry, Border irregularity, Colour irregularities, Darmascopic structures respectively of 
common pigmented skin lesions \citep*{LOESCHER2013170}.

\subsection{Support Vector Based Machine}
Thompson Felsia and Jeyakumar proposed research in 2017 on 
support vector machine based classifier to detect multi-lesions skin cancer by analysing pigmented skin lesions with an accuracy of 86.37 percent.
The proposed investigation with SVM based classifier has performed image segmentation using SRM (support region merging) algorithm. Furthermore, it employs SURF (speed up robust features) to find the region 
of interest for feature extraction to get optimal classification performance based on vector-based technique \citep*{thompson2017vector}. 
However, the research does not include image augmentation which generalises the predictions accurate to test in 
real-world environment. The research papers mention that support vector machine for automated classification of pigmented skin lesions is sensitive to the artefacts and can 
potentially increase the false positives which mean that predicted result for analysis was wrong positive prediction instead of an actual negative result. The investigation will perform image augmentation to generate random 
samples of images with different rotation angle and flipped images will be used to train and test the model to generalise the overall performance.

\subsection{Border Detection Based System}