The images with RGB(Red, Green and Blue) channels information was stored in NumPy multi-dimensional array with numbers ranging from 0 to 255.
The NumPy array was converted into the float32 format, and each element of the array was divided by 255 to normalise the data so, that it only ranges between 0 and 1 in float format which will help while training the model. In addition, one hot encoding was performed on class labels of the pigmented lesions. The one-hot encoding is a representation of the categorical variable as binary vector and normalises the categorical labels into a binary vector. One hot encoding was performed to the labels data for each pigmented skin lesions. The data normalisation process will help to train the models as images 
from training and testing data samples are ranging from 0 to 1 instead of 255 which results in 
the improved model accuracy.