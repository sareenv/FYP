The images with RGB(Red, Green and Blue) channels information was stored in numpy mutli-demensional array with numbers ranging from 0 to 255.
The numpy array was converted into the float32 format and each element of the array was divided by 255 to normalise the data so, that 
it only ranges between 0 and 1 in float format which will help while training the model. In addition, one hot encoding 
was performed on class labels of the pigmented lesions. The one hot encoding is a representation of categorical variable 
as binary vector and normalise the categorical labels into binary vector. One hot encoding was performed to the labels data for each 
pigmented skin leisons.