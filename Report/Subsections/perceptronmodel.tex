In this section, you should describe the problem that you set out to solve with the project. An introduction might, for example, begin by stating, "The aim of the work described in this Report was to provide a software tool with which people can arrange meetings." Avoid starting a Report with an irrelevant history of information technology. For example, the following would not be a good introductory sentence, "Since Bill Gates launched Outlook people have been using technology to arrange meetings."
Explain whatever background the reader will need in order to understand the problem. The background might refer to previous work in the academic literature that provides evidence that the problem is a real and significant problem worth solving. The background may identify a community, organisation or set of users that will benefit from your research. Include a clear and detailed statement of the project aims and provide an overview of the structure of the solution.
Explain whatever background the reader will need in order to understand the problem. The background might refer to previous work in the academic literature that provides evidence that the problem is a real and significant problem worth solving. The background may identify a community, organisation or set of users that will benefit from your research. Include a clear and detailed statement of the project aims and provide an overview of the structure of the solution.
CRITICAL! Use the introduction to define any terms or jargon that you will be using throughout the rest of the report.  Why?  Because people define and understand terms differently from one another.  Your definition of ‘cloud computing’ may be different to your supervisor’s definition of ‘cloud computing’.  By stating your definition clearly you can avoid misunderstandings of your work.
Conventionally, the last part of the introduction outlines the remainder of the Report, explaining what comes in each section – keep this brief.