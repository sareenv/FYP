The study is an attempt to use deep learning techniques in the classification of common pigmented skin lesions to detect skin cancer. Melanoma is serious skin cancer, and early diagnosis of such tumours can lead to the treatment of such skin problems at it’s early stages. The objective of the research is to develop an artificial intelligence-based automated system for detection of pigmented skin lesions for people with less mobility. The study involves experiments with different architectures of convolutional neural networks and understanding the effects of hyper-parameters on developing deep learning model. The results obtained from tests performed during the development of intelligent models were compared with predicted resulted in lesions by medical professionals.
