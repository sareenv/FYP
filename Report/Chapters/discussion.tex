\section{Achievements of Research Project}
The study has succeeded in developing a deep learning-based solution to detect skin cancers from pigmented skin lesions.
Different model experiments were performed to analyse the impact of architecture and hyperparameters on the accuracy of the model to identify the problem. 
The dataset was divided into separate testing data samples which were used for only evaluation purpose. The current research purpose model with the highest accuracy of 81.56\% performed on 11,00 data samples of pigmented skin lesions. 
The comparison was performed on the diagnosis by an automated system and medical professional to understand the time efficiency and reliability of the automated system. 
The results obtained from performing the comparison has shown that the automated system is significant time-efficient to perform diagnosis. 
In addition, the model has performed diagnosis to detect skin cancer from dermatoscopic images of pigmented skin lesion better than medical practitioners in some cases.
Furthermore, the intelligent model was also deployed on the web-based system to be used by the general audience. 
The research has helped me understanding the functioning of convolutional neural networks in visual recognition of the pigmented skin lesions
and conducting research while considering participants privacy.

\section{Deficiencies of Research Project}
The primary deficiency of the research is the limited collection of the responses for secondary research of the project. 
The widespread of pandemic coronavirus, has affected data collection from medical institutes ensuring the safety of the author of this research project. 
However, the small number of datasamples were collected from ten medical professionals to compare model performance.
The research experiments were constraint by the available hardware resources of 8 gigabytes of random access memory as result images of lower size and dimensions were used for image preprocessing and normalisation.

\section{Future Improvement}
The results obtained from the research can be used to understand the accurate hyper-parameters for detecting pigmented skin lesions. 
The current research project can further be improved by analysing the impact of applying the UNET semantic segmentation model which can result in potentially result in better accuracy of the model performance.
At last, the web system can be improved by developing the medical appointment booking system based on availabilty of medical professionals.
