The information was read using pandas into the data frame, 
which is a data structure that allows storing tabular data from CSV files. 
Datasets contain unclear and hairy images of pigmented skin lesions which were manually 
removed from the dataset to enhance the quality of available data.
The CSV file contained irrelevant information such as gender and age of 
patients and unbalanced data classes data columns were dropped from the dataset. Furthermore, the dataset was divided into training and 
testing sets using \url{sklearn.model_selection.train_test_split} class in the portion of 80 per cent for 
the training dataset and 20 per cent of testing datasets. The next step towards to preparing the dataset was reading the images data into NumPy 
array for both training and testing datasets and converting the image names from pandas series to NumPy array corresponding to each image and assign class number 
based on category of pigmented skin lesion in the dataset. Furthermore, the training and testing datasets were serialised into 
dictionary in a pickle encoded file. Therefore, the encoded file sizes are compact and are portable
in comparison to storing actual image files.